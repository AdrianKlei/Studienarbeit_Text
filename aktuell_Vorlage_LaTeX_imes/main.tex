% LaTeX Vorlage für studentische Arbeiten am imes, MZH, ...
% Vorlagenversion vom 15.08.2017
%
%
%
%
%
% META & VOREINSTELLUNGEN -------------------------------------------------------
\newcommand{\Autor}%                % Name des Autors / der Autorin
{Adrian Kleimeier}
\newcommand{\Matrikelnummer}%       % Matrikelnummer
{3111950}
\newcommand{\Datum}%       % Abgabedatum
{15. Januar 2021}

% ------------------------------------------------------------------------------------------
% erlaubt das Anfangen des Draft-Modus und damit Veränderungen
% einzustellen
\newif\ifdraft
% Draft-Modus: Arbeitsversion, Bilder werden nur als Rahmen dargestellt
% Vorteil: schnelleres Kompilieren
%\drafttrue % <- dafür hier die Kommentierung wegnehmen

%-------------------------------------------------------------------------------------------
% folgende Befehle sorgen dafür, das für jede Schrift die richtigen Pakete installiert werden
\newcounter{schrift}
% Schrift auswählen:
% 1 - mathptmx
% 2 - minionpro
% 3 - mathpazo
% 4 - times + computer modern
% 5 - computer modern
\setcounter{schrift}{4} % Hier die entsprechende Zahl setzen !

% ------------------------------------------------------------------------------------------
\input{Vorlagen/praeambel}                  % Präambel zur Dokumentenformatierung einfügen
                                            % braucht nicht zu verändert werden, stehen aber nützliche
                                            % Hinweise drin

\bibliography{Vorlagen/masterbib.bib}

%---------------------------------------------------------------
% Trennmuster fuer Ausnahmefälle
%---------------------------------------------------------------
\hyphenation{} % z.B. \hypenation{Trenn-text}
\hyphenation{Kraft-ein-wirk-ung}
\hyphenation{ein-ge-spannt}
\hyphenation{Coch-lea-im-plan-ta-te}
\hyphenation{Deh-nungs-mess-streifen}

%---------------------------------------------------------------
% Dokumentenanfang:
%---------------------------------------------------------------
% Einstellungen für PDF-Latex:
% Hier Titel etc. eintragen, dann wird es in den Dokumenteneigenschaften vom PDF richtig angezeigt
\ifpdf
    \hypersetup{
    %   colorlinks,  % Links mit farbigem Text
        pdftitle    = {zeit- und Ortsabhängige Prädiktion von Umweltzuständen für die Vorhersage von Personen- Auftrittswahrscheinlichkeiten},
        pdfsubject  = {Studienarbeit},
        pdfauthor   = {\Autor},
        pdfkeywords = {Kraftsensorik, Insertionskräfte, Cochleaimplantate}
        }
\else
\fi

\usepackage{Vorlagen/befehle}                        % eigene Befehle wie:
                                            % \zB\,
                                            % \abbildung{Position h,b,t,p}{Dateiname ohne Endung}{Caption}
                                            % \bild{Dateiname}, referenziert gemäß ...Bild x.xx...
                                            % einfach mal nachschauen was so drin ist und eigene Befehle für
                                            % wiederholende Sachen definieren

\includeonly{grundlagen}                   % wenn nur ein Kapitel kompilliert werden soll
                                            % geht schneller wenn man nur das Layout des Kapitels sehen will

%\entwurf                                   % Entwurfsdatum auf jede Seite setzen,
                                            % nicht vergessen beim Druck rauszunehmen
%\setlength\overfullrule{5pt}                % Overfull boxes werden angezeigt
%\setlength\underfullrule{5pt} 

% Seitenspiegel neu berechnen
\typearea[current]{last}
\begin{document}                            % Dokumentenanfang
\begin{spacing}{1.15}                       % Zeilenabstand auf 1,15 fach stellen, ist nicht so eng aber
                                            % auch nicht so eine Seitenschinderei wie 1,5-fach
    \frontmatter                            % mit kleinen römischen Seitenzahlen beginnen für class scrbook
    %\pagenumbering{roman}                  % das gleiche fücrartcl
    \setcounter{page}{0}
    \pdfbookmark[0]{Titel}{tit}             % damit der Titel auch im Acrobat angezeigt wird
		\begin{titlepage}
\begin{spacing}{2}

\begin{flushright} %rechtsbündig (Anfang)
	\vspace*{-20mm}
	\includegraphics[width=\textwidth]{Abbildungen/CoverLogos}
	%\includegraphics[width=\textwidth]{skizzen/CoverLogos_MZH}
\end{flushright} %rechtsbündig (Anfang)

% der Titel der Arbeit:
\vspace{38mm} {\centering {{\LARGE{Zeit- und Ortsabhängige Prädiktion von Umweltzuständen für die Vorhersage von Personen- Auftrittswahrscheinlichkeiten}}}

\vfill
% hier kommt eine hübsche Grafik hin:
\includegraphics[width = 80mm]{Titelbild_Bender.jpg}


\vfill }
\end{spacing}
\begin{spacing}{1}
\begin{tabular}{l}
 \Large{Studienarbeit Januar/2021}
% Die einzutragende Nummer gibt es beim Betreuer!
\end{tabular}

\vspace{5mm}

\begin{tabular}{l}
\large{\Autor}\\
\large{Matrikelnummer \Matrikelnummer}
\end{tabular}

\vspace{5mm}

\begin{tabular}{l}
\large{Hannover, \Datum}
\end{tabular}


\vspace{5mm}
{\large
\begin{tabular}{l l}
Erstprüfer  & Prof. Dr.-Ing. Tobias Ortmaier\\
Zweitprüfer & Prof. Dr.-Ing. Vorname Nachname\\
Betreuer    & Marvin Stüde, M.Sc.\\
\end{tabular}
}

\end{spacing}
\end{titlepage}
\cleardoublepage %Titelseite
		\include{Vorlagen/selbst} % Selbstständigkeitserklärung
    \clearpage
		%\includepdf{Vorlagen/aufgabenstellung.pdf} % Aufgabenstellung als pdf einbinden
		\newpage
\pdfbookmark[0]{Vortext}{frontmatter}
     \begin{flushright}
          \vspace*{-20mm}
          \includegraphics[width=\textwidth]{Abbildungen/CoverLogos.pdf}
     \end{flushright}
\vspace*{10mm} % mit \usepackage{printlen}: 1.8cm * \uselengthunit{cm}\printlength{\textwidth} / 16.01cm = -1.773cm -> passt nicht, 10mm durch optik
{\let\clearpage\relax}\thispagestyle{empty}\pdfbookmark[1]{Aufgabenstellung}{aufgabenstellung}

\begin{spacing}{1}        
\begin{center}
\large\textbf{Zeit- und Ortsabhängige Prädiktion von Umweltzuständen für die Vorhersage von Personen- Auftrittswahrscheinlichkeiten}

\normalsize \Autor, Matrikelnummer \Matrikelnummer

\end{center}

\textbf{Allgemeines:}

Am Campus Maschinenbau in Garbsen wird der Serviceroboter Sobi entwickelt, welcher Studenten und Besuchern Informationen bereitstellen und bei der Orientierung auf dem Campusgelände helfen soll. Um eine möglichst hohe Rate an Mensch-Roboter-Interaktionen zu erreichen, muss der Roboter Informationen über das zeit- und ortsabhängige Personenaufkommen auf dem Campusgelände zur Verfügung haben. Diese Informationen können für die Bahnplanung verwendet werden, um die voraussichtlich benötigte Zeit einer Kontaktaufnahme mit einem Menschen zu minimieren. In der Forschung existieren Modelle zur Prädiktion dieser Umweltzustände auf Basis von binären sowie quantitativen Darstellungen.

\bigskip\textbf{Aufgabe:}

Im Rahmen dieser Arbeit soll eine Methodik entwickelt und für den spezifischen Anwendungsfall am Maschinenbau-Campus Garbsen angepasst werden. Als Ansatz einer solchen Methodik kann das FreMEn- Modell dienen. Die Grundidee der Methode ist die Transformation binärer zeit- und ortsabhängiger Darstellungen elementarer Umweltzustände (in diesem Fall die Anwesenheit bzw. Nichtanwesenheit von Personen) in den Frequenzbereich mittels der Fouriertransformation. In diesem werden die dominantesten Frequenzen identifiziert und auf Basis dieser eine inverse Fouriertransformation durchgeführt. Als Ergebnis erhält man eine Funktion, welche die Wahrscheinlichkeit für die zukünftigen Umweltzustände angibt. \\

Im Rahmen dieser Arbeit ergeben sich insbesondere die folgenden Aufgabenpunkte:

\begin{itemize}
	\item{Literaturrecherche über Methoden der binären sowie quantitativen Darstellung von Umweltzuständen}
	\item{Entwicklung einer Methodik und Anpassung an den vorliegenden Anwendungsfall}
	\item{Überprüfung der Modellgüte bei unterschiedlichen Periodendauern mittels eines Testdatensatzes}
	\item{Überprüfung der Modellgüte bei unterschiedlichen Periodendauern mittels eines Simulations- Datensatzes oder eines Langzeitdatensatzes}
	\item{Ermittlung weiterführender Fragestellungen zur Optimierung der kurzfristigen Vorhersagegüte des Modells}
\end{itemize}

Die Bearbeitungszeit beträgt 300 Stunden.

\vfill
\begin{center}
\begin{tabular}{p{0.35\textwidth} p{0.35\textwidth}}
Ausgabe der Aufgabenstellung:&15.06.2020\\
Abgabe der Arbeit spätestens am:&15.01.2021\\
Erstprüfer: \\
Zweitprüfer: \\
Betreuer: M. Sc. Marvin Stüde
\end{tabular}
\end{center}

\end{spacing}
% Leerseite einfügen
\cleardoublepage % Aufgabenstellung aus LaTeX nutzen
		\include{Vorlagen/kurzfassung}
    \setcounter{page}{1}
    \pdfbookmark[0]{\contentsname}{toc}
    \tableofcontents                        % Inhaltsverzeichnis
    %\listoffigures
    %\listoftables
		\input{Vorlagen/nomenklatur}                    % Beschreibungen der verwendeten Abkürzungen und Variablen
    \mainmatter                             % der eigentliche Tex
    \setcounter{page}{1} % bei i anfangen
    \input{Vorlagen/einleitung}										% die einzelnen Kapitel einbinden
    	\chapter{Grundlagen}
% Mal schauen, ob man die Bilder noch besser einfügen kann hier
% Und bei jedem Bild die referenzierte Quelle angeben
In diesem Kapitel werden die mathematischen Grundlagen zum Verständnis der Arbeit vorgestellt. Während in Abschnitt \ref{sec:Fourierreihen und Fouriertransformation} auf die Darstellung periodischer Funktionen mittels Fourierreihen eingegangen wird, behandelt Abschnitt \ref{sec.Diskrete Fouriertransformation} die mathematische Formulierung der diskreten Fouriertransformation für abgetastete Signale und geht des Weiteren auf das Nyquist-Shannon-Abtasttheorem ein.\\
In Abschnitt \ref{sec.Wahrscheinlichkeitsverteilungen} folgt dann eine Erläuterung zu Wahrscheinlichkeitsverteilungen. Den Anfang bildet hier die Binomialverteilung (Abschnitt \ref{sec.Binomialverteilung}), bevor zuletzt die Poisson-Verteilung (Abschnitt \ref{sec.Poissonverteilung}) besprochen wird.

\section{Fourierreihen und Fouriertransformation}
% Quelle: Eichler 2006
% Warum werden hier die Fourierreihen mit Sinus-Termen beschrieben, aber später mit Cosinus-Termen?
% Konsistente Formelschreibweise: rho_n und a_n müssen einheitlich geschrieben werden
\label{sec:Fourierreihen und Fouriertransformation}
Periodische Signale tauchen in vielen Bereichen der Physik und Technik auf. Ein Signal bezeichnet hierbei eine Funktion, welche eine physikalische Größe in Abhängigkeit von der Zeit, dem Ort, oder einer anderen Variablen darstellt. Betrachtet man periodische Funktionen, so zeichnen sich diese durch ihre Periodendauer $T$ aus. Die gesamten Informationen des Signals dabei in einer Periode, so dass gilt: $f(t) = f(t+T)$. Jede periodische Funktion kann durch eine Überlagerung von Sinus- und Kosinusfunktionen unterschiedlicher Periodendauern $2 \pi n$ approximiert werden. Dargestellt werden kann dies durch eine Fourierreihe: 

\begin{equation}
	\label{eq:Fourierreihe}
	f(t) = \dfrac{a_0}{2} + \sum_{n=0}^N[a_n \cos(n \omega t) + b_n \sin(n \omega t)]
\end{equation}
Hierbei bezeichnet $\omega = 2 \pi / T$ die Kreisfrequenz der Grundschwingung. Im Allgemeinen geht $N \to \infty$. Die Konstanten $a_0,a_1 \dots a_n$ werden als gerade Fourierkoeffizienten bezeichnet, $b_1, b_2 \dots b_n$ hingegen als ungerade Fourierkoeffizienten. Dies leitet sich daraus ab, dass der Cosinus eine gerade und der Sinus eine ungerade Funktion ist. \\
Des Weiteren lässt sich eine Fourierreihe durch Sinusfunktionen mit unterschiedlichen Amplituden und Phasen beschreiben. Die Fourierreihe lautet dann:
\begin{equation}
	\label{eq:Fourierreihe_umgerechnet}
	f(t) = \rho_0 + \sum_{n=1}^{N} \rho_n \sin(n \omega t + \varphi_n)
\end{equation}
Die Grundfrequenz des Signals lautet $f_1 = 1/ T = \omega / 2 \pi$. Die weiteren Sinus- und Kosinusfunktionen der Fourierreihe besitzen Frequenzen $f_n = nf_1$, also ganzzahlige Vielfache der Grundfrequenz. Die Umrechnung von Gleichung \ref{eq:Fourierreihe} nach Gleichung \ref{eq:Fourierreihe_umgerechnet} erfolgt mithilfe der Definitionen:
\begin{equation}
	\label{eq:rho_0}
	\rho_0 = \frac{a_0}{2}
\end{equation}
\begin{equation}
	\label{eq:rho_n}
	\rho_n = \sqrt{a_n^2 + b_n^2}
\end{equation}
\begin{equation}
	\label{eq:phi_n}
	\Phi_n = \arctan(\frac{a_n}{b_n})
\end{equation}
Mit $\rho_0$ (\ref{eq:rho_0}) wird hierbei der Gleichanteil des Signals bezeichnet, $\rho_n$ (\ref{eq:rho_n}) steht für die Amplitude der \textit{n-ten} Frequenz und $\varphi_n$ (\ref{eq:phi_n}) für die Phasenverschiebung der \textit{n-ten} Frequenz. Ein Signal kann also durch sein Kosinus- und Sinusspektrum sowie durch sein Amplituden- und Phasenspektrum charakterisiert werden. 
% hier muss jetzt noch das Bild 52.1 eingefügt werden, welches das zeitliche Signal beschreibt, sowie Bild 52.2 mit dem Amplituden-und Phasenspektrum. Dann noch Erläuterungen dazu machen.
Grafisch veranschaulicht wird die Fourierreihe durch \bild{Fourier}. In der linken Grafik eingezeichnet ist ein periodisches Signal mit der Periodendauer T, für welches also $f(t) = f(t+T)$ gilt. In den beiden rechten Grafiken finden sich die das Gesamtsignal definierenden Frequenzen, mit ihren zugehörigen Amplituden $\rho$ und Phasenversätzen $\varphi$.
\begin{figure}[!ht]
	\centering
	\subfigure[]{\includegraphics[height=50mm]{Abbildungen/grundlagen/signal}}
	\hspace{5mm}
	\subfigure[]{\includegraphics[height=50mm]{Abbildungen/grundlagen/amplituden_phasenspektrum}}
	\caption{Periodisches Signal mit zugehörigem Amplituden- und Phasenspektrum (Quelle: \cite{Eichler.2006}))}
	\label{fig.Fourier}
\end{figure}
\\
Die Fouriertransformation überführt die Gleichung aus dem Zeitbereich $f(t)$ nun in den Frequenzbereich $F(\omega)$. Für analoge Signale ist die Fouriertransformation nun wie folgt definiert:
\begin{equation}
	\label{eq:Fouriertransformation}
	F(\omega) = \int\limits_{-\infty}^{\infty} x(t) e^\ind{-j\omega} \ind{d}t.
\end{equation}
\begin{equation}
	\label{eq:Inverse_Fouriertransformation}
	f(t) = \frac{1}{2\pi}\int\limits_{-\infty}^{\infty} F(\omega)e^\ind{j\omega} \ind{d}\omega
\end{equation}
Gleichung \ref{eq:Fouriertransformation} überführt eine von der Zeit abhängige Funktion aus dem Zeitbereich über Integration von $-\infty$ bis $\infty$ über alle Zeitpunkte $t$ in den Frequenzbereich. Die inverse Fouriertransformation wird durch Gleichung \ref{eq:Inverse_Fouriertransformation} beschrieben. Die Rücktransformation erfolgt durch Integration von $-\infty$ bis $\infty$ des von der Frequenz $\omega$ abhängigen Signals über alle Frequenzen $\omega$ \cite{Eichler.2006}.

\section{Diskrete Fouriertransformation}
\label{sec.Diskrete Fouriertransformation}
% Quelle: Wendemuth
Die in Abschnitt \ref{sec:Fourierreihen und Fouriertransformation} dargestellten Gleichungen gelten für analoge Funktionen. In der Praxis ist es aber so, dass keine vollständige Kenntnis über ein Signal vorliegt, sondern dies nur durch Messungen zu diskreten Zeitpunkten abgetastet werden kann. Hieraus resultiert die Diskrete Fourier-Transformation (DFT). Die resultierenden Werte $F(n)$ der diskreten Fouriertransformation eines zu den Zeitpunkten $k$ abgetasteten Signals $f(t)$ können mittels Gleichung \ref{eq:DFT} berechnet werden. Die inverse diskrete Fouriertransformation erfolgt dann durch Gleichung \ref{eq:IDFT}.
\begin{equation}
	\label{eq:DFT}
	F(n) = \sum_{k=0}^{N-1}x[k]e^\ind{j\frac{2\pi}{N}kn}
\end{equation}
\begin{equation}
	\label{eq:IDFT}
	f[k] = \frac{1}{N}\sum_{k=0}^{N-1}F(n)e^\ind{j\frac{2\pi}{N}kn}
\end{equation} \\
Im Zusammenhang mit der Diskreten Fouriertransformation ist abschließend das Abtasttheorem von Nyquist und Shannon zu nennen. Das Theorem besagt, dass ein beliebig geformtes, kontinuierliches Signal immer dann durch ein diskretes Signal darstellbar und auch exakt wiederherstellbar ist, wenn die Abtastfrequenz des Signals mindestens doppelt so hoch ist, wie die höchste im kontinuierlichen Signal enthaltene Frequenz. Beträgt die höchste Frequenz in unserem Signal also beispielsweise \SI{10}{\Hz}, so müssen wir es mit mindestens \SI{20}{\Hz} abtasten, um das Signal vollständig rekonstruieren zu können  \cite{Wendemuth.2005}. \\
Die Folgen einer zu geringen Abtastfrequenz werden in \bild{Abtasttheorem} ersichtlich. Die Abtastung des Signals zu den mit schwarz markierten Zeitpunkten reicht nicht aus, um das in grau dargestellte Originalsignal zu rekonstruieren. Stattdessen ergibt sich das in rot dargestellte Signal.
% Diese Abbildung nochmal mit Inkscape nachbauen
\begin{figure}[!ht]
	\begin{center}
		\includegraphics[height=50mm]{Abbildungen/grundlagen/abtasttheorem}
		\caption{Originalsignal (grau) und durch Abtastung rekonstruiertes Signal (rot) Quelle(https://www.geothermie.de/bibliothek/lexikon-der-geothermie/a/abtasttheorem.html)}
		\label{fig.Abtasttheorem}
	\end{center}
\end{figure}
\section{Wahrscheinlichkeitsverteilungen}
\label{sec.Wahrscheinlichkeitsverteilungen}
\subsection{Binomialverteilung}
\label{sec.Binomialverteilung}
% Quelle: Teschl 2010
Als Bernoulli-Experiment wird ein Zufallsexperiment bezeichnet, bei dem es lediglich zwei Ausgänge geben kann. Ein Ereignis A tritt entweder ein oder nicht. Führt man ein Bernoulli-Experiment n-mal hintereinander unter den gleichen Bedingungen durch, so erhält man eine Bernoulli-Kette der Länge $n$. Das Eintreten des Ereignisses A wird gemeinhin als Erfolg bezeichnet, die Wahrscheinlichkeit $P(A)=p$ bezeichnet man als Erfolgswahrscheinlichkeit. Als Ereignis A kann hier beispielhaft das Werfen einer Münze mit dem Ausgang $Zahl$ genannt werden. Eine Binomialverteilung entsteht nun, wenn wir die Anzahl der Erfolge bei einer Bernoulli-Kette ermitteln wollen. Mathematisch formuliert lässt sich die Binomialverteilung ausdrücken als:
\begin{equation}
	\label{eq:Binomialverteilung}
	P(X=x) = \binom{n}{x}p^\ind{x} q^\ind{n-x}
\end{equation}\\
$X$ bezeichnet hierbei die Anzahl der Versuchsdurchführungen, bei denen ein Erfolg eintritt. $X$ kann die Werte $x = 0,1,2,\dots,n$ annehmen, $p$ steht für die Eintrittswahrscheinlichkeit eines Erfolges, $q$ für die Wahrscheinlichkeit eines Misserfolges. Die Zufallsvariable $X$ ist binomialverteilt und ihre Wahrscheinlichkeitsverteilung eine Binomialverteilung mit den Parametern $n,p$. Kurz: $X \sim Bi(n;p)$. Die grafische Darstellung einer Binomialverteilung für die Wahrscheinlichkeit der Anzahl an Würfen eines Würfels mit dem Ereignis \textit{1} bei sieben Würfen ist in \bild{Poisson_Verteilung} dargestellt \cite{Teschl.2014}.
% Hier Bild aus Buch 2014 Mathe für Informatiker einfügen
% evtl noch Eigenschaften der Binomialverteilung einfügen? Oder unnötig?

\begin{figure}[!ht]
	\begin{center}
		\includegraphics[height=50mm]{Abbildungen/grundlagen/Binomialverteilung}
		\caption{Binomialverteilung mit $n = 7, p = \frac{1}{6}$ Quelle \cite{Teschl.2014}}
		\label{fig.Poisson_Verteilung}
	\end{center}
\end{figure}
\subsection{Poissonverteilung}
\label{sec.Poissonverteilung}
Eine Zufallsvariable $X$, welche unendlich viele Werte $x=0,1,2\dots$ mit den Wahrscheinlichkeiten
\begin{align}
	P(X=x) &= \frac{\lambda^\ind{x}}{x!} e^\ind{-\lambda} & (\lambda >0)
	\label{eq:Poisson Verteilung}
\end{align}
annehmen kann, wird als poissonverteilt mit dem Parameter $\lambda$ bezeichnet. Die zugehörige Verteilung heißt Poisson-Verteilung. Der Erwartungswert sowie die Varianz der Poisson-Verteilung werden ausgedrückt als:
\begin{align}
	\mu &= E(X) = \lambda \\
	\sigma^2 &= Var(X) = \lambda
	\label{eq:Poisson EW und Var}
\end{align}
Anhand der obigen Formeln erkennt man, dass der Parameter der Poisson-Verteilung grade gleich ihres Erwartungswertes ist, selbiges gilt für die Varianz.
Häufig ist es von Interesse, die Anzahl $X_t$ eines Ereignisses innerhalb eines Zeitraumes $[0, t]$ zu prognostizieren. Die Menge von Zufallsvariablen $X_t, t\geq 0,$ wird als Poisson-Prozess mit der Intensität $\lambda$ bezeichnet, falls $X_t$ einer Poisson-Verteilung folgt, es also gilt: \\
\begin{equation}
	P(X_t=x) = \frac{(\lambda t)^x}{x!} e^\ind{-\lambda t}
	\label{eq:Poisson-Prozess}
\end{equation}
Ein Poisson-Prozess muss dabei nach \cite{Teschl.2014} drei Voraussetzungen erfüllen:
\begin{itemize}
	\item Die Wahrscheinlichkeit für ein Ereignis ist proportional zur Beobachtungsdauer $\Delta t$, aber unabhängig von der Lage des Beobachtungsintervalls.
	\item Die Wahrscheinlichkeiten für ein Ereignis an unterschiedlichen Orten sind voneinander unabhängig
	\item Für infinitesimal kleine $\Delta t$ ist die Wahrscheinlichkeit, dass das Ereignis mehr als einmal auftritt, im Vergleich zur Wahrscheinlichkeit, dass es genau einmal vorkommt, vernachlässigbar klein \cite{Teschl.2014}.
\end{itemize}

% Vllt noch eine Sektion zu GMM's (Gaussian Mixture Models) ? Wird in Paper Krajnik.2015b aufgegriffen. Evtl. Erklärung da etwas zu kurz gegriffen.



    	\chapter{Stand der Technik}

In diesem Kapitel wird auf den Stand der Technik eingegangen. Im weitesten zitiere ich hier die wissenschaftlichen Paper, auf denen meine Arbeit beruht. Ob man das Kapitel noch weiter einteilen muss, wird sich später zeigen, erstmal lasse ich es ohne Unterkapitel. \\
Mobile Roboter finden immer mehr Einzug in Umgebungen, welche von Menschen bewohnt sind. Diese Menschen üben Aktivitäten aus, welche in der Folge zu Veränderungen eben dieser Umgebung führen. Man kann davon ausgehen, dass viele dieser Aktivitäten täglichen Routinen mit typischen Mustern folgen, welche von mobilen Robotern erkannt werden und zur robusteren Darstellung ihrer Umgebung genutzt werden können. Mapping in statischen Umgebungen stellt ein weit erforschtes Gebiet dar \cite{Eichler.2006}. Für das Mapping in dynamischen Umgebungen gibt es verschiedene Ansätze. Während ein Ansatz darauf abzielt, sich bewegende Objekte aus der Umgebungsdarstellung herauszufiltern \cite{Hahnel.30Sept.5Oct.2002}, werden in anderen diese Objekte getrackt und als bewegte Landmarken klassifiziert \cite{Montesano.2008}. Diese separations-basierten Ansätze können jedoch nicht auf Langzeitveränderungen der Umgebungsstruktur eingehen. \\
Im Gegensatz hierzu stehen adaptive Ansätze, welche davon ausgehen, dass die Karte niemals komplett ist und diese durch kontinuierliches Mapping aktualisieren. So können der Karte durch neue Observierungen des mobilen Roboters neue Features hinzugefügt werden 
\cite{Milford.2010}. In \cite{Krajnik.2014} wird nun erstmalig versucht, die räumlich-zeitliche Dynamik der Umgebung durch ihr Frequenzspektrum darzustellen. Die Zustände von lokalen Umgebungsmodellen, wie zum Beispiel einer Tür, welche entweder offen oder geschlossen sein kann, sollen hier durch Wahrscheinlichkeitsfunktionen repräsentiert werden, welche aus der Superposition periodischer Funktionen entstehen. In \cite{Krajnik.2014} wird als Motivation dazu angeführt, dass die meisten Mapping Ansätze wichtige Komponenten der Umwelt, wie z.B. eine Tür, durch lediglich zwei eindeutige Zustände dargestellt werden. Eine Tür ist also entweder geöffnet oder geschlossen. % Hier noch eine entsprechende Quelle einfügen.
Diese Zustände können jedoch auch durch ihre Wahrscheinlichkeit $p_j$ ausgedrückt werden. Bayes-Filter gehen hierzu von einen statischen Welt aus, d.h. die Wahrscheinlichkeiten der Zustände $p_j$ werden als konstant angesehen. Durch neue Beobachtungen können diese konstanten Annahmen verändert werden, alte Beobachtungen werden so jedoch über die Zeit "vergessen" \cite{Krajnik.2014}. Nimmt man jetzt jedoch an, dass diese Zustandswahrscheinlichkeiten Funktionen der Zeit sind, also $p_j (t)}$ gilt, und diesen zeitlichen Veränderungen der Wahrscheinlichkeiten eine finite Nummer periodischer Prozesse zu Grunde liegt, könnte man den Einfluss und die Periodizität eben dieser Prozesse identifizieren und die Zustandswahrscheinlichkeit $p_j (t)$ aus dieser Beschreibung ermitteln. In \cite{Krajnik.2014} wird nun die in Abschnitt \ref{sec:Fourierreihen und Fouriertransformation} erläuterte Fouriertransformation benutzt, um diese periodischen Prozesse zu identifizieren. Als Beispiel wird ein Belegungsnetz herangeführt. Jede der Zellen des Belegungsnetzes kann zwei Zustände $s_j = \{frei, belegt\}$ annehmen. Diese Zustände sind jedoch nicht konstant, sondern eine Funktion der Zeit, also $s_j (t)$. Die Unsicherheit des Zustandes wird nun durch sein Wahrscheinlichkeit $p_j (t)$ ausgedrückt. Da die Zellen unabhängig voneinander sind, kann die Fouriertransformation separat auf jede Zelle des Belegungsnetzes angewendet werden. \\ Die über die Zeit aufgetragenen Zustände einer Zelle $s(t)$ werden mittels der Fouriertransformation $P = FT(s(t))$ transformiert. Es werden l Koeffizienten $P_i$ des Spektrums $P$ ausgewählt und zusammen mit ihren Frequenzen $\omega_i$ benutzt, um mittels der inversen Fouriertransformation $p(t) = IFT(s(t))$ die Wahrscheinlichkeitsfunktion $p(t)$ des Zellzustandes zu bestimmen. Abschließend wird ein Schwellwert benutzt, um aus $p(t)$ eine Schätzung $s'(t)$ der tatsächlichen Zustandsfunktion $s(t)$ zu bestimmen. Das Set $P$ besteht hierbei aus $l$ Tripeln mit den Einträgen $(abs(P_i), arg(P_i), \omega_i)$, wobei $abs(P_i)$ für die Amplitude, $arg(P_i)$ für den Phasenversatz und $\omega_i$ für die Frequenz des jeweiligen periodischen Prozesses steht, welcher den Zustand $s(t)$ beeinflusst. \\

Der Zustand einer Zelle wird nun über die Gleichung xy approximiert. 
\begin{equation}
	s(t) = (IFT(P) \gtr 0.5) \oplus  (t \notin 0)
	\label{eq: Zellzustand}
\end{equation}
Ist die Wahrscheinlichkeit $p(t)$ einer Zellbelegung größer als 0.5, so wird die Zelle als belegt geschätzt, sofern der Zeitpunkt $t$ nicht zum Set der Ausreißer 0 gehört.
%Ausreißer-Set muss noch eingebunden werden.
Der in Gleichung benutzte Schwellwert von 0.5 kann willkürlich gesetzt werden. So können Vorhersagen über zukünftige Zustände der Zelle mit einem gewissen Konfidenzniveau von $c$ durch die Gleichung:
\begin{equation}
	s'(t,c) = IFT(P) > c
	\label{eq:Zustandsvorhersage}
\end{equation}
getroffen werden. Grafisch verdeutlicht wird die Methodik durch \bild{FreMEn Beispiel}. 

\begin{figure}[!ht]
	\begin{center}
		\includegraphics[]{example_of_measured_state_and_prediction}
		\caption{Beispiel eines über die Zeit gemessenen Zellzustandes sowie seines Spektralmodells und Wahrscheinlichkeitsprädiktion Quelle (Krajnik.2014)}
		\label{fig.FreMEn Beispiel}
	\end{center}
\end{figure}
In der linken Grafik rot dargestellt sind die über einen zeitlichen Verlauf aufgenommenen, binären Zustände einer Beispielzelle. Der grüne Graph beschreibt das zugehörige FreMEn-Modell der Ordnung drei. In blau aufgetragen sind die Vorhersagen des Modells ermittelt anhand eines Schwellwertes von 0.5. Der lila Graph stellt die Zeitpunkte dar, zu denen die Modellvorhersage von den tatsächlichen Zellzuständen abweicht \cite{Krajnik.2014}. Die rechte obere Grafik repräsentiert das Frequenzspektrum der Zelle, die für das Modell ausgewählten Frequenzen sind durch blaue Kreise markiert. Das zuvor erwähnte Tripel bestehend aus Amplitude, Phasenversatz und Frequenz der jeweiligen periodischen Prozesse ist in der rechten unteren Grafik dargestellt. Um die Auswirkungen des Modellgrades, also der Anzahl der in das Modell einfliessenden periodischen Prozesse, zu erforschen, wurde die Methodik auf einen Datensatz angewendet, bei welchem ein SCITOS-G5 mobiler Roboter ausgestattet mit RGB-D und Lasersensoren, Personen in einem Bürogebäude über eine Dauer von einer Woche mit einer Rate von \SI{30}{\hertz} detektiert hat. \\
Die Genauigkeit des Modells $q(t_a,t_b)$ wird anhand von Gleichung xy berechnet und beschreibt das Verhältnis von korrekt geschätzten Zellzuständen zu der Gesamtdauer des betrachteten Intervalls.
\begin{equation}
	q(t_a,t_b) = \frac{1}{t_b - t_a} \int_{t_a}^{t_b} |s'(t) - s(t)| \ind{d}t
\end{equation}
Unterschieden wurde nun in \cite{Krajnik.2014} zwischen dem Rekonstruktionsfehler $q_r$ sowie dem Prädiktionsfehler $q_p$. Der Rekonstruktionsfehler beschreibt, wie genau das Modell Zeitintervalle beschreibt, welche zur Ermittlung der Modellparameter verwendet wurden. Der Prädiktionsfehler hingegen beschreibt die Genauigkeit des Modells in Bezug auf Zeiträume, welche nicht zur Modellermittlung verwendet wurden. Die ermittelte Abhängigkeit des Rekonstruktions-sowie Prädiktionsfehlers von der Modellordnung ist in \bild{Modellgenauigkeit} aufgezeigt. \\
\begin{figure}[!ht]
	\begin{center}
		\includegraphics[]{Prediction_Reconstruction_Error}
		\caption{Modellgenauigkeit vs. Modellordnung Quelle (Krajnik.2014)}
		\label{fig.Modellgenauigkeit}
	\end{center}
	
\end{figure}

Die Rekonstruktionsgenauigkeit liegt bei einer Modellordnung von 15, d.h. es wurden 15 periodische Prozesse zum Approximieren des Zustandssignales verwendet, bei 95 \%. Die Rekonstruktionsgenauigkeit $q_r$ steigt dabei monoton mit der Modellordnung, die Prädiktionsgenauigkeit $q_p$ hingegen nicht. % Manchmal nenne ich q_p hier Prädiktionsgenauigkeit und manchmal Fehler, das muss ich noch einheitlich machen. 
\\
Die lokalen Maxima von $q_\ind{p1}$ und $q_\ind{p2}$ lassen den Schluss zu, dass für die Vorhersage eine Modellordnung von zwei oder drei optimal ist (siehe \bild{Modellgenauigkeit}).


		\input{Vorlagen/tipps}
		\input{Vorlagen/vorlage_betreuer}
		\chapter{(Beispielkapitel) Robotersysteme}

Das folgende Kapitel soll als Beispielkapitel dienen. Nach der Kapitelüberschrift wird der Kapitelinhalt in ein paar Sätzen beschrieben.
Hier steht weiterer Text.

\section{PR2}

Der PR2 (Willow Garage Inc., Menlo Park, USA) ist ein \textbf{\textit{menschenähnlicher}} Serviceroboter, der seinen Dienst in Wohnräumen verrichten soll und derzeit im sogenannten PR2 Beta-Programm von elf Forschungseinrichtungen über einen Zeitraum von zwei Jahren getestet wird 
\cite{WillowGarage2010}.
Hier steht weiterer Text.  d

\begin{figure}[!ht]
	\centering
	\subfigure[]{\includegraphics[height=50mm]{PR2-1}}
	\hspace{5mm}
	\subfigure[]{\includegraphics[height=50mm]{PR2-2}}
	\hspace{5mm}
	\subfigure[]{\includegraphics[height=50mm]{PR2-3}}
	\caption{Serviceroboter PR2 von Willow Garage (Quelle: Willow Garage))}
	\label{fig.PR2}
\end{figure}

Ausgestattet ist der PR2 mit zwei Armen, die jeweils sieben Freiheitsgrade haben und an deren Enden ein Greifer montiert ist, siehe \bild{PR2}. Die Sensorik des Armes besteht aus einer Kamera am Unterarm und Druck- sowie Beschleunigungssensoren am Greifer. Die Nutzlast eines Arms ist mit \SI{1,8}{kg} ausgewiesen. Weiterhin verfügt der Roboter über einen dreh- und schwenkbaren Kopf, in dem eine 5-Megapixel Farbkamera, ein LED-Texturprojektor und zwei Stereokameras integriert sind, wobei eine Kamera für die Fernsicht und die andere für die Objektmanipulation genutzt wird. Unterhalb des Kopfes ist ein schwenkbarer Laserscanner und ein Inertialsensor verbaut. Die Position des Oberkörpers lässt sich in der Höhe zwischen \SI{1330}{mm} und \SI{1645}{mm} (Gesamthöhe) variieren. Angetrieben wird die omnidirektionale Basis von vier gelenkten Rädern, die eine maximale Geschwindigkeit von \SI{3,6}{\kilo\metre\per\hour} ermöglichen. Die quadratische Basis hat eine Kantenlänge von \SI{668}{mm}. Als Recheneinheit stehen zwei Server zur Verfügung, die jeweils auf acht CPU-Kernen rechnen und dabei auf 24\,GB Arbeitsspeicher zugreifen können. Als Betriebssystem wird Ubuntu verwendet, auf dem das Robot Operating System, kurz ROS, die Grundlage für die Datenverarbeitung bildet. Da ROS innerhalb dieser Arbeit ebenfalls zum Einsatz kommt, wird dieses in \Sec{ROS} vorgestellt und an den entsprechenden Stellen weiter erläutert. Die Kosten für einen PR2-Roboter belaufen sich derzeit auf etwa \SI{400 000}{\text{US-Dollar}}\footnotemark. Mit Hilfe des PR2 wurden von den zuvor erwähnten Beta-Testern Szenarien bewältigt, die innerhalb des menschlichen Wohnraumes auftreten können. An der TU München hat ein PR2-Roboter beispielsweise zusammen mit einem anderen Robotersystem einen Pfannkuchen gebacken \cite{TUM2011}.
\footnotetext{Der angegebene Preis wurde am 16.08.2011 der Website http://www.willowgarage.com/pages/pr2/order entnommen und versteht sich exklusive Steuern und Versandkosten.}

\section{ROS}
\label{sec.ROS}

Hier steht weiterer Text.

\begin{table}
	\centering
	\caption{Technische Daten der youBot Plattform}\label{tab.TechSpecYouBotBase}
	\vspace*{-3mm}
	\begin{tabular}{lcr}
        \toprule
		Bezeichnung		& Formelzeichen	&              \\
		\midrule
		Gesamtlänge 	& $a$           & \SI{530}{mm} \\
		\rowcolor{Snow2}
		Gesamtbreite 	& $b$           & \SI{350}{mm} \\
		Höhe			& $h$           & \SI{106}{mm} \\
		Radstand		& $l$           & \SI{470}{mm} \\
		\bottomrule
	\end{tabular} 
\end{table}

\begin{lstlisting}[label=source.launchHokuyo,caption=Launchfile zum Start der hokuyo\_node]
<!-- launch hokuyo node -->
<node pkg="hokuyo_node" type="hokuyo_node" name="hokuyo_node" output="screen">
	<param name="port" value="/dev/ttyACM0"/>
	<param name="frame_id" value="/base_laser_front_link"/>
</node>
\end{lstlisting}
    \nocite{*}                             % alle Literaturquellen einbinden, sonst werden nur die zitierten
                                            % Quellen im Literaturverzeichnis angezeigt (ist Geschmackssache).
                                            % eher nicht verwenden, außer man hat einen guten Grund
    \appendix                               % Anhang starten, jedes weitere Kapitel bekommt jetzt einen Buchstaben
    \chapter*{Anhang}                       % Anhang als Chapter
    \addcontentsline{toc}{chapter}{Anhang}
    %\thispagestyle{empty}                   % keine Kopfzeile, Seitenzahl u.a., leere Seite mit Überschrift Anhang
    %\setcounter{chapter}{1}                 % Chapter Counter auf 1 = im Anhang A
    %\setcounter{equation}{0}                % Equation Counter nullen
    %\newpage                                
    %\ihead{\normalfont Anhang}              % Kopfzeile auf Anhang setzen



    %% --- Ab hier der Anhang einfügen

    %\include{anhang_wheatstone}            % Anhang
    %\include{anhang_fehlerfortpflanzung}
		%\include{anhang_mgcEinstellungen}
		%\include{anhang_trafos}
		%\include{anhang_befestigen}
		%\include{anhang_datenblaetter}
    %% --- Anhang zu Ende
		
    \ihead{\normalfont\headmark}            % kolumnentitel innen
 
    %% --- Literaturverzeichnis
    {\sloppy \printbibliography}

\end{spacing}
\end{document}                              % fertig!

