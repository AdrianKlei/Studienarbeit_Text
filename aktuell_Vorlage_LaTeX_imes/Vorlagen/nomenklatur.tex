%---------------------------------------------------------------------------------------------------------------------------
\chapter*{Nomenklatur}
\pdfbookmark{Nomenklatur}{}
%---------------------------------------------------------------------------------------------------------------------------
Selten bzw. nur abschnittsweise verwendete Symbole und Formelzeichen sowie abweichende Bedeutungen werden ausschließlich im Text beschrieben. \textbf{Achtung: Bitte bei Erstellung der Arbeit die unten stehenden Beispiele löschen und nur Abkürzungen/Zeichen aufführen, die verwendet werden!}
%---------------------------------------------------------------------------------------------------------------------------
\subsubsection{Allgemeine Konventionen}\vspace{-3mm}
%---------------------------------------------------------------------------------------------------------------------------
\begin{tabbing}
	1234567890 \= \kill
	Skalar \> Klein- oder Großbuchstabe (kursiv): $a$, $A$    \\
	Vektor \> Kleinbuchstabe (fett und kursiv): $\vec{a}$     \\
	Matrix \> Großbuchstabe (fett und kursiv): $\vec{A}$      \\
	Punkt  \> Klein- oder Großbuchstabe: $\ind{a}$, $\ind{A}$ \\
	Körper \> Großbuchstabe (fett): \textbf{A}              
\end{tabbing}
%---------------------------------------------------------------------------------------------------------------------------
\subsubsection{Lateinische Buchstaben}\vspace{-3mm}
%---------------------------------------------------------------------------------------------------------------------------
\begin{tabbing}
	1234567890 \= \kill
	%------------------------------------------------------------------------------------------------------------------------
	$A$																		\> Querschnittsfläche \\
	$A_\ind{S}$                           \> Spanungsquerschnitt \\
	und so weiter
\end{tabbing}
%---------------------------------------------------------------------------------------------------------------------------
\subsubsection{Griechische Buchstaben}\vspace{-3mm}
%---------------------------------------------------------------------------------------------------------------------------
\begin{tabbing}
	1234567890 \= \kill
	$\alpha, \ \beta, \ \gamma$												  \> Rotationswinkel um die $x$"~, $y$"= und $z$"=Achse                                         \\	
\end{tabbing}
%---------------------------------------------------------------------------------------------------------------------------
\subsubsection{Koordinatensysteme}\vspace{-3mm}
%---------------------------------------------------------------------------------------------------------------------------
\begin{tabbing}
	1234567890 \= \kill
	$(\ind{KS})_{i}$							        \> Koordinatensystem $i$ \\
	$\ks{0}$							        				\> ortsfestes Inertialkoordinatensystem \\
\end{tabbing}
%---------------------------------------------------------------------------------------------------------------------------
\subsubsection{Abkürzungen}\vspace{-3mm}
%---------------------------------------------------------------------------------------------------------------------------
\begin{tabbing}
	1234567890 \= \kill
	AR			\> erweiterte Realität (Augmented Reality) \\
	CNC     \> rechnergestützte numerische Steuerung (Computerized Numerical Control) \\
	MHH     \> Medizinische Hochschule Hannover \\
\end{tabbing}
